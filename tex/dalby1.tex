\documentclass[a4paper,twocolumn,swedish,11pt]{article}
\usepackage[utf8]{inputenc}
\usepackage[T1]{fontenc}
\usepackage{times}
\usepackage{babel}
\usepackage{url}

\addtolength{\oddsidemargin}{-1cm}
\addtolength{\textwidth}{25mm}

\begin{document}
\begin{enumerate}
	\item Allmänt hälsotillstånd

		{\footnotesize mycket gott, gott, \fbox{\textbf{någorlunda}}, dåligt, mycket dåligt}

	\item EQ-5D: {\footnotesize \fbox{\textbf{XXX}}~poäng}

		{\footnotesize\begin{enumerate}
			\item Rörlighet:

			\item Hygien:

			\item Huvudsakliga aktiviteter:

			\item Smärtor/besvär:

			\item Oro/nedstämdhet:
		\end{enumerate}}

	\item Läkemedel/preparat

		{\footnotesize\begin{enumerate}
			\item Ja, senaste tre månaderna:
			\item Ja, senaste 14 dagarna:
		\end{enumerate}}

	\item Besvär/symtom

		{\footnotesize\begin{enumerate}
			\item Värk i skuldror, nacke eller axlar:
			\item Ryggsmärtor, ryggvärk, höftsmärtor eller ischias:
			\item Värk eller smärtor i händer, armbågar, ben eller knän:
			\item Ängslan, oro eller ångest:
			\item Astma, allergi:
		\end{enumerate}}

	\item Stress

		{\footnotesize Ja, ofta; Ja, ibland; \fbox{\textbf{Nej (nästan aldrig)}}}

	\item Sömn

		{\footnotesize mycket bra, ganska bra, \fbox{\textbf{varken bra/dålig}}, ganska dålig, mycket dålig}

	\item Göra själv för att bevara en god hälsa

	\item Förändringsbenägenhet: {\footnotesize\fbox{\textbf{5: Vidmakthållande}}}

	\item Fysisk träning: {\footnotesize \fbox{\textbf{XX}}~minuter/vecka}

	\item Vardagsmotion: {\footnotesize \fbox{\textbf{XX}}~minuter/vecka}

	\item Fysisk aktivitet under åren:

		{\footnotesize\begin{enumerate}
			\item Mestadels stillasittande, ibland någon promenad eller liknande
			\item Lättare fysisk aktivitet, såsom gång, cykling, trädgårdsarbete, minst 1--2~timmar i veckan
			\item Mer ansträngande fysisk aktivitet, såsom motionslöpning, gympa, bollsport, minst 1--2~timmar i veckan
			\item Hård träning eller tävling, regelbundet och flera gånger i veckan.
		\end{enumerate}

		\begin{center}
			\begin{tabular}{lcccc}
				Ålder&1&2&3&4\\\hline
				20--29&&&&\\
				30--39&&&&\\
				40--49&&&&\\
				50--59&&&&\\
				60--69&&&&\\
				70--  &&&&
			\end{tabular}
		\end{center}}

	\item Fysisk aktivitet, senaste sju dagarna (IPAQ):

		{\footnotesize\begin{description}
			\item[mycket ansträngande]

			\item[måttligt ansträngande]

			\item[promenader]

			\item[sittande]
		\end{description}}

	\item Har hund, promenader:

	\item Matvanor: {\footnotesize \fbox{\textbf{XX}}~poäng}

	\item Tobak:

	\item Alkohol:

		{\footnotesize\begin{enumerate}
			\item \fbox{\textbf{XX}} standardglas en vanlig vecka
			\item \fbox{\textbf{XX}} standardglas eller mer vid ett och samma tillfälle
		\end{enumerate}}

	\item Vardagliga situationer

		{\footnotesize\begin{enumerate}
			\item ASI: \fbox{\textbf{XX}}~poäng

			\item Self afficacy: \fbox{\textbf{XX}}~poäng
		\end{enumerate}}

	\item Ångest: {\footnotesize \fbox{\textbf{XX}}~poäng}; Oro: {\footnotesize \fbox{\textbf{XX}}~poäng}

	\item Fysisk aktivitet och motion

		{\footnotesize\begin{enumerate}
			\item Inre motivation: poäng

			\item Yttre motivation: poäng

			\item Amotivation: poäng

			\item Förmåga: poäng

			\item Stöd: poäng
		\end{enumerate}}
\end{enumerate}
\end{document}

